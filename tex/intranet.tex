\section{内网资源}

\subsection{校园网}

BUPT的校园网也许不是最好的,但一定是最便宜的(指完全免费)。

北邮的几个校区\footnote{海南校区的情况暂且不明}共用超过60Gbps的聚合出口带宽\footnote{移动40Gbps,电信20Gbps,CERNET 1Gbps,CERNET2 10Gbps},一般来说你的设备的网速都可以跑满无线或者有线连接本身的带宽,据称,即使在最高峰的时候,出口带宽也从未用满过(不过不要指望这会使你抢课的时候拥有什么优势)。

沙河校区的建筑室内都有WiFi的覆盖,室外在楼宇附近也有覆盖。在校园内覆盖了无线网络的区域,一般可以搜索到BUPT-mobile、BUPT-portal等SSID。BUPT-mobile使用WPA2-Enterprise(802.11x)加密和认证,BUPT-portal使用Captive Portal认证。BUPT-mobile关闭了2.4Ghz频段,因此通常推荐连接BUPT-mobile以避开跳频等问题,获得更好的上网体验。

在校外必须使用学校的VPN才能访问内网的资源。可以使用\href{https://vpn.bupt.edu.cn/}{aTrust}(推荐),或者bug多多兼容性堪忧的\href{https://webvpn.bupt.edu.cn/}{WebVPN}。

\subsubsection*{网关账号和密码}

师生在校期间拥有以学工号为账号的网关账号。这一账号独立于统一认证及教务系统的账号,用于登入校园网、VPN等系统。可以在\href{https://netaccount.bupt.edu.cn/}{校园网自服务系统}查看账户状态及修改密码。

\emph{注意:绝对不要把自己的校园网账号借与他人!网关账号密码直接关系到连接内网的权限,联网操作若造成破坏均由账号所有人承担责任!}

\subsubsection*{连接到BUPT-Mobile}

首先下载并安装北邮校园网用于本地网络认证的\href{https://github.com/FredericDT/BUPTCampusNetworkManual/blob/master/Wireless/XTC-BUPT-mobile-assets/BUPT-Local-Server-Certificate.crt}{证书},命名为“BUPT Local Server Certificate”并信任该证书。

然后,连接到BUPT-mobile。连接时,在CA证书一栏选择刚刚安装的“BUPT Local Server Certificate”,域名也填写“BUPT Local Server Certificate”,EAP方法选择“PEAP”,阶段2身份验证选择“MSCHAPV2”,身份和密码分别填写学号和网关账号密码,点击连接即可。

\subsubsection*{连接到BUPT-Portal}

一些设备出于种种原因可能无法连接到BUPT-mobile,这时可以使用BUPT-portal。

连接到BUPT-portal后,访问任一plain HTTP网站如\href{http://www.msftconnecttest.com/redirect}{msft}即可跳转到认证页面,输入网关账号密码即可。

\subsection{常用网站}

\subsubsection*{可以直接访问的网站}
\begin{itemize}
    \item \href{https://www.bupt.edu.cn/}{北邮官网}:北京邮电大学的官方网站。
    \item \href{https://webvpn.bupt.edu.cn/}{WebVPN系统}:使用浏览器访问内网资源,无需安装客户端和插件,支持电脑和手机直接使用。
    \item \href{https://vpn.bupt.edu.cn/}{aTrust}:校园网VPN,可以下载客户端,相比于WebVPN使用更加便捷,登陆密码为企业微信实时更新的一串数字。
    \item \href{https://ucloud.bupt.edu.cn/}{云邮教学空间}:新开通的一个教学平台,部分课程会在里面发放资料、提交作业甚至考试。根据反馈,这个平台还有很多bug,不少同学被坑。
    \item \href{https://libcon.bupt.edu.cn/}{电子资源统一访问系统}:WebVPN套皮,可以在校外访问学校购买的IEEE、ACM、SCI等索引数据库。
\end{itemize}

\subsubsection*{需要通过VPN或在校园网环境下访问的内网资源}
\begin{itemize}
    \item \href{http://my.bupt.edu.cn/}{信息门户}:发布校内通知、新闻等,也整合了各个网站系统的入口。勤看信息门户有助于不错过重要通知。
    \item \href{https://jwgl.bupt.edu.cn/}{本科生教务系统}:查课表考表成绩学分、抢课退补选等等,都在教务系统。
    \item \href{https://service.bupt.edu.cn/}{网上服务大厅}:线上办理各种手续的地方。\sout{甚至可以在线申请退学}
    \item \href{https://iclass.bupt.edu.cn/}{“爱课堂”教学平台}:部分科目的老师用来进行线上分发教学资料、收发作业和在线考试的平台。
\end{itemize}

\subsubsection*{其他相关的网络平台}
\begin{itemize}
    \item \href{https://mail.bupt.edu.cn/}{北邮邮箱}:开学后可以注册,适合用来做你自己学术专用的邮箱。带有@bupt.edu.cn后缀的邮箱被称为教育邮箱,可以用来申请很多对学生免费的资源。按教育部要求,教育邮箱会在毕业半年后被收回,转为@bupt.cn后缀的校友邮箱。
    \item \href{https://bbs.byr.cn/}{北邮人论坛}:最早的中文校园BBS之一,可以注册一个账号与其他同学交♂流,具体见下。
    \item \href{https://byr.pt/}{BYRBT}:知名的PT站,只有使用非国内三大运营商的IPv6地址才能访问,具体见下。
    \item \href{http://tv.byr.cn/show}{北邮人IPTV}:卫星电视直播站,可以看到很多台的电视直播,有的校园活动也会在上面转播,不过好像不太稳定(摊手
\end{itemize}

\subsubsection*{学校使用的一些公共网站平台}
\begin{itemize}
    \item \href{https://www.yuketang.cn/web}{雨课堂}:清华开发的平台,少部分的(现在几乎没有了)课程课件会发布在这里,可以使用微信登录。
    \item \href{https://www.ketangpai.com/}{课堂派}:部分老师收发作业用的平台,疫情期间的部分考试也使用此平台来发放、提交试卷。使用体验不算太好,有时会遇到卡顿。可以使用微信登录。
\end{itemize}
以上平台有微信小程序。
\begin{itemize}
    \item \href{https://passport.zhihuishu.com/}{智慧树}:网课平台之一,比较良心。
    \item \href{http://erya.mooc.chaoxing.com/}{超星尔雅}:网课平台之二,臭名昭著。\footnote{在2022年6月中旬发生了拖库事件,超星的数据库被全部泄露出去了,造成了很多人的隐私泄露}
    \item \href{https://www.icourse163.org/learn}{中国大学MOOC}:疫情期间才引入的网课平台,只有部分老师在里面开课。不过上面有很多其他大学的老师开放的在线课程,可以拓展一下知识面或作为课堂内容的补充。
\end{itemize}
以上平台有自己的App。

\subsection*{常见问题}

\faq{如何利用学校提供的学术资源?}

学校的图书馆系统中有“数据库导航”链接,由此可以进入查看学校所购买的学术资源数据库,包括(国内的)万方、知网和(国外的)IEL、Nature等。在校内(使用校园网的情况下),你可以直接进入这些网站并下载所需要的资料;在校外,你需要选择校外登录等方式,通过登录认证从而领用上述资源。以知网为例:

\begin{center}
    \includegraphics[width=0.4\textwidth]{images/cnki-login.png}
\end{center}

在校内,直接使用IP登录即可;在校外,需要点击校外访问,选择北京邮电大学,用校园网账号登录即可。在学校购买的数据库范围内下载论文、资料是全部免费的,学校每年给这些机构支付了天价的使用费,大家不需要自己花钱。

\faq{如何注册北邮人论坛?有什么用?}

使用你的北邮校园网帐号注册即可,一个学生号最多能注册三个账号,要经过手机验证后才能发帖。除了直接访问网站,也可以下载北邮人论坛的App,安卓、苹果都有,但体验不算太好。北邮人论坛是校内最重要的论坛,有学校老师入驻以确保言论合法且不具有攻击性。你可以在这里看到保研资讯、校内新闻、经验技术分享等等,是获取校内信息的重要途径。

\faq{BYRBT是什么?如何使用?}

BitTorrent是一种资源分享协议,它的宗旨是“我为人人,人人为我”。在BT资源站上,你可以下载到很多其他人分享的高质量资源(包括但不限于软件、电影、音乐、游戏、图书资料等——需要注意的是,这些资源并不一定都是正版),但同时你应当履行保存这些资源并利用你的网络上传给他人分享的义务。如果你只下载资源而不提供上传分享,或者随意在网站外分享得到的资源,BT站有权利封禁你的账号。

在符合要求的网络环境(例如校园网)下,就可以登录BYRBT。一个学号一般只能注册一个BT账号,所以一定要珍惜你的账号,不要因为滥用下载和随意外传资源而被封号。账号长时间不使用也会被删除(这个功能暂时没有开启),但是你可以手动“封存”它,在放假或者长时间不在学校前记得先封存账号,回到学校后解封它。

BYRBT是一个PT站,PT(Private Tracker)是一种基于私有BT Tracker服务器的资源传播形式,经授权的用户使用受允许的客户端进行种子制作与下载。要使用BYRPT提供的服务,就必须遵守站内的相关规则。注册并登录后你可以在网站上看到规则和常见问题,请仔细阅读。

\faq{我为什么上不了网/我为什么连不上校园网/为什么校园网这么卡}

你可以使用\href{https://buptnet.icu}{这个平台}进行自助诊断(仅限内网访问),以及通过企业微信-信息帮和62283039报修电话联系信息化技术中心。也同样可以加入网络反馈QQ群进行反馈,群号是835973564。
