\section{住宿条件}

\subsection{沙河宿舍简介}

根据宿舍楼的位置,沙河校区的宿舍分为雁北园和雁南园。相对而言,雁北园的条件会差一些,比如阳台空间较小、公共浴室较为破旧和储物空间较少。但其实也只是差一点(

\emph{雁北园}位于鸿雁路北侧,A/B/C/D1是四栋矩形相连的宿舍楼(一楼入口独立,二楼以上的部分相连),D2/E是另两栋独立于A/B/C/D1宿舍楼且相对较小的宿舍楼,内部也是相连的。离食堂操场生活区较近。雁北园各区域的代号为A/B/C/D1/D2/E。

\emph{雁南园}位于鸿雁路南侧,S2 S3 S4 S5是四栋平行的宿舍楼,六层,离教学楼和景观湖(线程池?)较近。南区食堂也修好了,即将投入使用。S6是一栋单独的宿舍楼,由于2020年才投入使用,所以条件是全校最好的。雁南园各区域的代号为S2/S3/S4/S5/S6(S1是信息中心楼)。

学校有三栋教学用楼宇的代号也是S1(教学楼)、S2(学院楼)和S3(实验楼),不要和宿舍楼弄混哦。S2-S5在物理上连通,S6为单独一栋楼。男女宿舍边界处的门是锁住的。按照学校的安排,所有的女生应该会统一入住S4、S5和S6,其中S4为男女混住,其余宿舍均为男生宿舍。

\subsection{宿舍环境}

无论是雁北园还是雁南园,宿舍的基本配置都是:四人间,上床下桌,有独立卫生间。卫生间内只有一个坑位和洗脸面盆,不能洗澡(但有地漏)。有的宿舍楼的独立卫生间会分成两个部分,分别是坑位和洗脸池。厕所里没有垃圾桶,需要自己购买。房间内有空调和暖气片,有阳台。

\begin{center}
    \begin{minipage}{0.45\textwidth}
        \centerline{\sffamily\small 雁北和除S6以外的雁南宿舍}
        \centerline{\includegraphics[width=1\textwidth]{images/dorm.png}}
    \end{minipage}
    \qquad
    \begin{minipage}{0.45\textwidth}
        \centerline{\sffamily\small 雁南S6宿舍}
        \centerline{\includegraphics[width=1\textwidth]{images/dorm-s6.png}}
    \end{minipage}
\end{center}

\faq{我需要带什么生活用品吗?}

其实你可以什么也不带,报道当天都可以去学校超市购买。当然你自己带好常用的洗漱用品、床单被套啥的也可以。(非必须物品建议提前物流、网购或现场买,不建议千里迢迢携带。寄邮政到学活\footnote{学生活动中心一层的邮政}会离宿舍近一点。)

\faq{宿舍楼还有其他配置吗?}

每层楼有一个公共卫生间和澡堂(如果不想经常清理房间里的厕所,建议多去公共卫生间),一楼有宿管的值班室,可以使用微波炉。每栋楼有一个电梯,一楼有若干自动售货机(买饮料零食泡面啥的),支持移动支付。部分楼层会有空出的房间作为自习室。

\faq{澡堂是怎么样的?}

纯淋浴房。采用刷卡计费的方式,大概每分钟0.1元。澡堂中午12:00到晚上23:00开放(其实管的不严的情况下一般不锁门),建议最好还是提前去洗,晚上的时候人还是比较多的。实测全天有热水,多放一会儿即可。雁北E区、雁南S6的澡堂拥有隔板,其他澡堂暂无。

\faq{宿舍楼有门禁吗?}

没有,但有宵禁。宿舍楼开放时间是早上6点到晚上11:00,如果要十一点以后回宿舍,就要提前联系辅导员进楼。带外来人员进楼要在宿管处登记。沙河校区附近的治安并不算太好,建议大家不要在深夜独自外出。

\faq{水电费怎么算?}

水费目前没有收。电费每人每学期赠送40度电(所以宿舍四个人一共赠送160度)。电费价格:12元可购买25度电(只有空调最花钱,要不然花不了多少电)

\faq{上床下桌的尺寸是多大?}

\begin{center}
    \includegraphics[width=0.55\textwidth]{images/bed-size.png}
\end{center}

又到了搬出老图的时候——不同宿舍的柜子尺寸可能不同,上床离天花板大概一米多一点,桌子上可以放得下27"显示屏。

\faq{我太高了,床不够长怎么办?}

报到前可以提前申请更换加长床,同时购买加长的床垫(要求身高190cm以上),但缺点是大概率不会和同班(甚至同专业)的同学分在一个宿舍。

\subsection{宿舍生活}

\faq{我会和我的同班同学住一个寝室吗?}

一般会,但也可能和同专业的其他班同学住一起。极端情况下你会与其他专业的同学一个寝室。分寝室结果在报道前可以在\href{https://welcome.bupt.edu.cn/}{迎新系统}上查到。\footnote{宿舍通常由辅导员分配,开学前辅导员应该会把你拉进一个QQ通知群,如果有特殊需求可以单独沟通}

\faq{辅导员会检查宿舍吗?}

开学后会有宿舍评比,主要看装饰和整洁度,优秀宿舍会加德育分(关于德育分的问题详见后文),具体看学院政策。辅导员查寝的频率取决于你的导员有多懒,以及学院有没有发动什么相关运动。

每周会有宿管卫生检查(实际上没来过几次),按百分制进行计分,基准分为100,不符合规定的倒扣对应分数。

\faq{宿舍装饰有什么限制吗,可以拉床帘吗?}

因为一些安全上的考量,规定上不允许安装床帘,实际执行上不严。有部分宿管会限制蚊帐、床帘和地毯,具体购买之前可以先询问宿管和辅导员。

\faq{宿舍限电吗?晚上是否拉闸?}

沙河校区的宿舍都比较新,没有严格限制用电功率,规定不允许使用大功率电器和电热器具(具体限制可以参看贴在宿舍门上的须知,电吹风可以在指定的地方使用)。晚上不拉闸,二十四小时通电通网。

\faq{宿舍里有东西坏了找谁修?}

从宿舍里的空调到澡堂的热水器,基本上都可以直接打电话(后勤接诉即办,也可以在企业微信里找到在线应用)报修,具体报修电话可以问楼道的宿管。
