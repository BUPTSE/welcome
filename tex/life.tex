\section{生活相关}

\subsection{生活服务}

\faq{学校内有哪些生活服务?}

生活服务区在雁北园A座(靠近鸿雁路那栋)负一楼,有超市、打印店、文具礼品店、理发店、水果店、眼镜店和电子产品修理店,部分店价钱小贵但可以接受,支持刷学生卡、刷脸或移动支付。

提示:如果是重度水果患者,可以考虑点外卖购买水果或者干脆去沙河物美大采购,可能比在水果店购买水果便宜。

文具店的价格比较公道了,没必要另寻他路。超市里不定期会有打折商品和捆绑销售的商品,如果你能接受,那就是赚到。理发店价格20元一次普通理发,也可以染发什么的。如果觉得不好,可以去沙河镇上的理发店。(理发店的水平真是一言难尽

学生食堂对面有一家名为小麦铺的商店,商品较为齐全,里面还有一家打印店和一家水果店。在打印店打印时要注意资料安全,有需要保密的资料请慎重。除小麦铺与生活服务区的超市里的打印店外,图书馆一层也有自助打印机,各种二手群里也可以寻到不少同学提供价格优惠的打印服务。

学生活动中心一层内有邮政门面和运营商的营业厅,以及另一家电脑修理店。此类修理店价格都比较贵,如果你的电脑只是需要清灰或者加装硬盘可以找身边的电脑爱好者搞定,二手群里也有同学以相当低的价格提供类似服务,如果电脑硬件出现比较大的问题,建议联系原厂售后。

在各种新生群以及校园内你可能会遇到推销手机校园卡套餐的人,如果你确实需要的话,可以在各个群里多方打听一下,寻找一个价格相对较低的售卡代理。这种校园卡套餐只有新开卡时有优惠,续费只能原价,不推荐在使用期间用于注册或绑定各种账号。

\faq{校内有洗衣房吗?}

雁北D2地下和雁南S4地下均有洗衣房,自助洗衣,1元5分钟甩干,6元35分钟洗涤,7元45分钟洗涤。你也可以选择在卫生间洗手池那里手洗。但是不允许自己购买洗衣机(因为需要保证用电安全)。

\faq{收发快递方便吗?}

根据快递公司不同,你会遇到下面几种情况:

中国邮政:学校亲儿子,门面设在学生活动中心一楼,收发快件都很方便,很便宜,但是比较慢(也就慢1-2天),但是营业时间有限(9:00-16:00)。

其他快递:学校门口的菜鸟驿站,营业时间为早上九点到晚上八点。取件区域分为京东、顺丰与菜鸟。

京东区域取件只需要向工作人员提供你的4位手机尾号即可。顺丰和菜鸟快递都需要根据取件码寻找快递,上面会显示你的快递在几号柜几号架子上,然后出库。其中菜鸟快递出库需要使用菜鸟App的出库码,找到门口的机器上扫码出库。(可以用自己的取件码代取别人的快递)

菜鸟驿站春夏秋晚上八点关门,冬天晚上七点半关门,记得及时去取。


\faq{在校园里如何付钱?}

生活服务区的超市和小麦铺均支持微信付款码支付与刷校园卡,此外,生活服务区的超市也支持微信刷脸支付。

如果校园卡绑定了微信\footnote{在完美校园公众号中绑定},除澡堂与热水机外的所有刷校园卡支付机器均可以用微信扫码、微信付款码或完美校园付款码支付。\footnote{微信与完美校园付款码的区别在于,前者花微信中的钱,而后者所花的是校园卡中的钱。}

除此之外,大部分场所支持微信扫码支付


\subsection{餐饮美食}

\faq{沙河有几个食堂?伙食怎么样?}

一共有三个。两个食堂在北侧二维码广场,按建筑外墙上的名字,分别称为风味餐厅与教工餐厅,各有五层,不论学生老师都可以用餐(不要被教工餐厅这个名字给骗了,学生也可以去哦)。另一个食堂在学校最南边,叫作学生餐厅\footnote{沙河的食堂命名比较混乱,21级及以前的学长学姐大都还沿用“教”“学”“南”的俗称:\\\hspace*{4em}教工食堂:地图中标示为风味餐厅,A 楼\\\hspace*{4em}学生食堂:地图中标示为教工餐厅,D 楼\\\hspace*{4em}南区食堂:地图中标示为学生餐厅,南区食堂楼}。除了风味餐厅五层是点菜制的有包厢的餐厅(也是唯一可以使用移动支付和现金的餐厅),其他楼层都是窗口制。学生餐厅的窗口以基本伙为主。

每天的窗口快餐菜单可以关注餐厅公众号(沙邮餐饮)查询。

基本伙:教三 风二 风三

美味快餐:教一 教二 风一

美食城:教四 教五 风四 风五

注:教/风+数字代表教工餐厅/风味餐厅的第几层

此外还有一家咖啡厅位于图书馆一楼东侧(环境非常舒适,服务到位,很适合作为和同学聊天的地方,有时候也能看到日麻社的人在这里打麻将。牛腩饭还挺好吃的,不过价格有点小贵,但是经常会发半价券什么的,可以关注一下);以及一个面包房(面包八元一个有点小贵,店里时常有一些活动,还有个粉丝群,喜欢吃面包的可以向周围人打听一下)和一个西餐厅(实际卖汉堡薯条,披萨非常难吃,另外下午四点半之后还有烧烤),位于风一旁边。教学楼、学生活动中心、东配楼等都有自助咖啡机和自动售货机。

\faq{点外卖方便吗?}

学校在小南门\footnote{从西门进校后路的右侧(南侧)有一个小门,以前是一个栅栏铁门,现在拆除铁门改为了外卖柜}安装了两个外卖柜,学生点的外卖需要统一到外卖柜去取,从宿舍走个来回大约需要10分钟。

\faq{饮水方便吗?}

教学楼、图书馆、宿舍等楼层里面都有热水机,插学生卡即可使用,非常方便,但水垢味很重。寝室都可以不用热水瓶。有一个水站提供桶装水,也是一种选择。

\faq{有哪些一般人不知道的美食?}

水果店里的糖葫芦:5块钱一根还贼好吃(15块钱的草莓糖葫芦也非常美味),数量有限记得每天下午早点去蹲点。

风四烤鸭架:虽然没有标出来,但你可以不花\sout{16.8}\sout{17.8}18.8买烤鸭套餐而选择6块一份的鸭架骨,分量大骨头上的肉也很多,物超所值。

\subsection{活动场所}

\faq{图书馆的开放时间?借阅规则?}

图书馆服务区早上八点到晚上十点开放(节假日另行通知),刷学生卡进入,一次可以借阅两本书,时长30天,可续借两次(每次可续15天)。一切操作都在自动借还机上完成(放假时自动续借)。如果超时归还,会遭到一段时间不能借书的惩罚,时间过长还会被罚款。

图书馆一共五层,每层均有很多自习座位,一层为自习区,不受放假闭馆时的限制,7:00-22:50开放,学习环境超级棒,建议需要认真学习时都来图书馆。

图书馆二楼提供查询资料的临时电脑,免费。此外还有自助打印机,不过价格比打印店要贵。

图书馆五楼有个小型博物馆,里面有远古电脑等有趣的展品。

\faq{如果要小组讨论问题或者做大作业(或者和好基友闲聊),有哪些地方去?}

研讨间(推荐):图书馆的2-5楼均有,是环境很好的小隔间,可供2到8人进行研讨。需要在微信小程序上提前预约,预约的时候要选择使用时段(一次最多预约两个小时),并且必须有一人按时去打卡使用,否则会因为咕咕咕而受到惩罚(好像是三次违约后本学期不能预约)。因为房间有限,黄金时段很难抢到,建议必要的时候提前1天预约。环境较好,隔音能力中等(外面的人听不见,但是隔壁的人有可能听见)。部分研讨间设有大屏幕,可以连接电脑进行投影。如果需要大屏幕但你预约的研讨间里没有,你也可以和图书馆管理员协商一下,移过来一台。

教学楼教室:现在每栋教学楼的1楼都有智慧屏,可以查询教室的使用情况,找一个该教室没课的时候,大家都可以自由使用教室。大教室一般会有同学在里面自习,所以需要小组讨论,可以去N楼和S楼之间的小教室,教室周末也可以使用。S1教学楼的教室环境更好一些,有的还有移动白板可以使用,适合中等规模的研讨。如果需要举行会议或其他活动,可以通过学生组织或辅导员申请借用教室,在批准的借用时段可以独占整个教室。

塞纳左岸咖啡馆:在图书馆东侧一楼。环境很优雅,服务很周到,不过没有隔音,所以不要进行一些会很吵的活动,不过闲聊什么的倒是一个好去处。开放时间为早上9点到晚上10点。

风味五楼包间:在非就餐时间是没人的,餐厅也不会管在这里讨论的同学,还是比较推荐的。

学生活动中心:里面有很多各个部门专用的办公室、会议室,如果你加入了一些学生组织,可以联系当部长副部长的学长学姐,他们也许会“借”给你一个房间。因为不常有人来,所以环境非常安静,适合思考问题。

\faq{我想组织活动,哪里有场地?}

如果是小型活动,可以去学生活动中心的各个学生组织办公室,如果是大型活动,可以去申请教室,或者去申请学生活动中心的活动场地。申请场地可以请辅导员帮忙,也可以通过社团等学生组织申请。

在学生活动中心地下有一个健身房,健身房里有一个台球桌;在一层的空地有镜子,一些时候可以看到有人在此跳舞;二层有一个小舞台。在教学实验综合楼的N楼S楼之间地下的室外有一处羽毛球场地,S楼地下有很大的乒乓球场地,一共二十多张乒乓球桌。

\faq{还有什么特别的去处吗?}

特别推荐图书馆四楼的蛋壳椅,很适合一个人(或者两个人叠在一起)缩在里面自闭听歌!

\subsection{交通往来}

\faq{我可以在学校里面滑滑板吗?}

可以!都可以!宿舍门口和教学楼门口用黄黑胶带划定了区域放置滑板,不过没有人看管,要小心失窃哦。

\faq{学校离地铁站太远怎么办?}

走(毕竟就1km)!或者骑自行车,自行车可以停在沙河高教园地铁站附近,上一个车锁基本可以保证安全(不要停在沙河站,那里人多眼杂)。

美团单车也是很不错的选择,现在每天都会有人定时定点往学校里投放美团单车,地点位于操场和雁北园中间的马路,再也不会出现以前满校园找不到一辆共享单车的情况了。

\faq{到底往南走坐地铁方便,还是往北走方便?}

往南沙河站,往北沙河高教园站,距离基本一致。如果需要顺便购物,从沙河站下车附近有超市和餐厅。但是如果是高峰时段沙河站非常拥挤,提前一站坐车会舒服一点(别想了还是没有位子),\sout{并且相对人流涌动成分复杂的沙河站来说,沙河高教园站相对安全一些}不过沙河高教园站附近人烟稀少,经常有奇怪的人出没,这一带治安都一般,大家要注意人身和财产安全。

由于北京地铁昌平线采取大小交路运行,沙河高教园站会有始发车(空车!有座位!)。

\href{https://www.bjsubway.com/station/xltcx/linecp/2013-08-26/246.html?sk=1}{沙河高教园站列车时刻表}

\href{https://www.bjsubway.com/station/xltcx/linecp/2013-08-26/249.html?sk=1}{沙河站列车时刻表}
