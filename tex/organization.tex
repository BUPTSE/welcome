\section{学生组织}

\faq{沙河有哪些学生组织?怎么加入?}

粗略可分为校级、院级、社团三种。

\emph{校级}:校团委(含职能部门和直属组织)、校学生会和研究生会、鸿雁新闻媒体中心(校党委宣传部/新闻中心直属)、运动队等

% 校团委:办公室、组织部、宣传部、文体部、社团工作部、志愿者工作部(真情志愿者联合会)、科创实践部、研究室、北邮青年融媒体中心(北邮青年新媒体工作室、第二课堂项目办、北邮人论坛)、大学生文化素质教育中心(大学生艺术团);沙河校区团工委(内设办公室、组织部、宣传部)、学生组织学生社团团工委。
% 校学生会:办公室、学习部、宣传部、权益部、体育部、文艺部。

\emph{院级}:各个学院的团委、学生会、志愿者协会\footnote{比如计算机学院是星火志协}等

\emph{社团}: 所有学生社团,还包括曾经是校级组织的大学生艺术团(2021年起改为社团)。在信息门户和群文件中有完整白名单,可仔细查阅。

原则上团委领导和主管各种学生组织。学生会、研究生会在团委和学生处、研工部指导下开展工作,各社团在行政上受校团委社团工作部管辖。志愿者协会受校团委志愿者工作部管理。一些学生组织有多块牌子,同时是学生社团和学校行政单位的直属组织,对于一般同学来说没有必要深究。

% 学生社团和学生组织经常受到双重管辖:团委的行政管辖和指导单位的业务指导。

% 同时是竞赛队和行政直属单位:ICPC集训队/信息学中心
% 同时是社团和行政直属单位:大学生艺术团
% 同时是竞赛队和学生社团:RoboMaster队、智能车/机器人社,天枢CTF
% 具有以半官方名义组织或参加活动能力的社团:乒乓球协会等体育社团

% 竞赛队本不应该算作一种学生组织,但是像ICPC,CTF这样的竞赛队伍具有一定的组织形态,因此也列在这里。本质上受到学校直接支持开展参赛活动的竞赛队伍都是依托于行政单位或学生社团。

加入校级学生组织要困难一点:团委和团委直属组织(青梅、北邮人、志协)、学生会、鸿雁,他们都会在开学期间进行招新,需要进行一到两轮的面试才能加入,在高中有很强社会活动经验或者有一些技术上的特长会很有帮助。团委内设机构原则上要求必须是团员,直属组织无此要求。如果你不是,也可以开学的时候和导员申请入团,但大概率错过团委招新。

而加入院学生组织比较容易:院学生会也会统一组织招新,面试也比较简单,名额比较多。

如果这些面试你都挂了,还可以在班会的时候竞选班委,成功率也会很高。

加入社团基本没有限制:一般而言如果你对某个社团感兴趣可以直接加他们的群,如果想要成为社团的正式成员需要在学校的学工系统中正式申请加入,每位同学最多可以正式加入两个社团。而中心组成员一般需要在10月通过报名、面试等环节进行筛选。(一般而言中心组成员才视为社团干事)如果你进入了某个社团的中心组,就相当于进入了该社团的管理层,需要负责安排对应社团的事物。如果只是普通正式成员,是不算校级学生组织的。

社团中有一类较为特殊,即功能型社团,如大学生艺术团以及国旗护卫队等,这类功能性社团承担有学校级别的任务,因此不像兴趣类的社团那样可以随意加入,需要通过面试等环节才能加入,通常还有较高的专业水平要求。

% 大学生艺术团:管乐团、交响乐团、民乐团、爱乐合唱团、民舞团、话剧团、街舞团等)

\faq{学生组织会很忙吗?都有哪些福利?}

最重要的:在德育分中有一定比重的职务分。(但是多个职务累加还是取最高就不一定了)

所以如果你想要更高的综测成绩,可以至少加入一个院/校级学生组织。

因为招新进入的同学都是干事身份,所以大多组织的活其实是不多的,基本是每个月或两个月组织活动的时候才会有分配任务。校/院级组织的活动频率低,但要求高,需要你花点心思完成你的工作。而班委基本只需要完成日常工作和导员的小任务,难度低但会比较繁琐。

学生组织和班委的任职年限都是一年,一年后就可以选择放弃职务或留下一年或竞选职务,如果竞选成功会升职为副部长之类的职位,可以获得更高的德育分评定,但也会更加忙碌。每个部在大二末期还会选择部长,除了部长一般大三以后不再担任学生组织的职位。因此如果你前两年不参加,后面就没有办法加入学生组织了。

不用担心学生组织会影响考试复习,考试周期间一般是不会安排任务的。但如果你遇到了很扯淡的老师/部门(点名批评社团工作部,考试周还给社团派活),那就不好说了。

各系学生组织都会有各种福利,包括但不限于:不定期聚餐(公费吃喝的很少,AA制的更多)、为你过生日会、组织各类娱乐活动,瓜分多余的活动奖品等等。并且你可以比其他同学更早获得学校活动的信息,有名额限制的活动(比如大学间联谊,观影会、公益演出等)可以获得内部机会不用抢票之类。并且你会有同部门的学长学姐可以指点迷津。一些大型活动的名额一般还是会更加青睐学生组织成员的。

\faq{如何参加大创活动?有哪些好处?}

在大一、大二年级可以参与雏雁计划(“大创的练兵场“)体验大创流程。一般为10月立项,次年4月结题。此后可以正式参与大创,一般于7月公布立项结果,次年6月进行评定。中间会有若干中期评定等环节,更为正式。

雏雁或大创获奖会在德育分评定上有加成,未来找实习简历上也可以多一些东西。

\faq{学校还有哪些经常举办的活动?}

学校的公共活动:运动会、电影公映会(《我和我的祖国》、革命者、五个扑水的少年)、进校园演出(上一学年开心麻花、国家大剧院合唱团等来本部有过演出)、各种类型的讲座等等。

社团活动:秋季学期游园会(百团大战),春季学期百花节,均为社团大摆摊的形式,玩得痛快。

每个院组织的大型活动:一般有歌手比赛(计算机学院的秋之韵)、表演比赛、辩论赛等。

大型志愿活动:比如2019年的国庆游行方阵,2021年的建党100周年庆祝大会,首都举行的各种会议,包括冬奥会,都会在大学生中招募志愿者。
