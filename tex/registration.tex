\section{新生报道}

\subsection*{学号与班号}

首先欢迎2022级软件工程的同学们加入这个豪华孤儿院大家庭!
当你正式来到北邮的时候,便会得到两串十位数字:学号和班号。我们依次解读:
\begin{itemize}
    \kaishu
    \item 学号:2022(年份)21(全日制本科生)XXXX(四位学生编号)
    \item 班号:202121(同上)13(计算机学院\footnote{计算机学院(国家示范性软件学院)})16/17/18/19/20(软件工程的五个班\footnote{按照2021年的分班方法,这五个班以及一个双培班会成为四大班})
\end{itemize}

当你接触许许多多的新账号时,优先尝试账号为学号,密码为身份证后6/8位的组合,能解决80\%的问题。(注意,身份证号含有X的需要大写)

\subsection*{报道时间}

{\heiti 【待修改】开学时间节点(2021年)}
\begin{itemize}
    \kaishu
    \item 8月27日 报道
    \item 8月29日 9:30-11:00 2021级本科生英语入学分级及免修资格考试
    \item 8月30日-9月12日 沙河校区校内军训
    \item 9月13日 正式第一天授课
\end{itemize}

报道当天会有志愿者在北京站、北京南站、北京西站给坐火车的同学们接站,到学校以后也会有志愿者(就是学长们啦)全程接驾,千万不要紧张哦~
