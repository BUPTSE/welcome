\section{军训相关}

\faq{时间与地点}

由于疫情原因,近几年的军训都在校内举行(原本是在八达岭军训基地)。有消息称,从23级开始军训地点重归军训基地。另外,根据北京市的统一要求,从2022级起,军训时间由入学前调整至大一下学期结束后的暑假。

\faq{变动对比}

据小道消息,23级军训的时长增加到21天,加上极有可能重归军训基地训练,可谓是前所未有的艰苦。根据往届的说法来看,在军训基地14天只能体验3次限时铁链澡堂,以及每天餐饮仅有包子、鸡蛋、咸菜等粗茶淡饭。鉴于军训时间、军训地点以及军训时长等等因素全方位的改变,目前唯一有参考价值的是往年军事爱好者协会整理的军训生存指南\href{https://shimo.im/docs/473QMD0rGVc6MV3w/}{2018北邮新生军训求生攻略},上面详细地记录了军训基地的衣食住训四个环节,并附带有QA以及推荐物品附录。但鉴于年代久远,长期未得到维护与更新,无法保证能起到多大作用。

\faq{军训内容}

\emph{(无法保证有效性)}
\begin{itemize}
    \item 必修:齐步、踏步、正步等队列动作
    \item 选修:军体拳、擒敌拳、旗语、匕首操、应急棍术、格斗术、刺杀操、战术(随机分配)
    \item 活动:拉歌、看电影等
\end{itemize}
