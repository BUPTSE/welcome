\section{校区环境}

北邮现在使用中的校区一共三个,即西土城路校区(海淀区,本部)、沙河校区(昌平区)以及海南校区(仅限玛丽女王海南学院)\footnote{另有西城区小西天校区(校舍)和昌平区宏福校区(接近弃用),目前和大家关系不大,唯有实验课偶尔会去一两次。}。

校本部位于海淀区西土城路10号,面积很小,住宿条件相对较差,但交通便利,对面就是北师大,周围美食众多。目前,大部分大三大四的本科生及多数研究生在本部学习。沙河校区是新生入学的校区,位于昌平区沙河镇南丰路1号,规划面积很大(大概本部三倍)但是实际只建成规划面积的一半多;住宿学习环境好,但比较偏僻,出行不便。

除海南学院外的所有新生都将在沙河校区入学并在这至少读完大一。

\begin{center}
    \includegraphics[width=0.80\textwidth]{images/shahe-map.jpg}
\end{center}

\faq{校园到底有多大?绕一圈要多久?}

很小,真的很小。本部绕一圈仅需10分钟,沙河大概也就20分钟。{\small (不过小也有小的好处,比如说可以7:55起床去上8:00的课。)}

\faq{我什么时候会从沙河搬到本部?}

按照学校最新的安排,各个学院的本科生从沙河搬到本部的时间如下表:

\begin{center}
    \begin{tabular}{cc}
        \toprule
        学院 & 搬迁时间(学年开始) \\
        \midrule
        计算机学院(国家示范性软件学院) & 大二 \\
        经济管理学院 & 大二 \\
        国际学院 & 大二\\
        信息与通信工程学院 & 大三 \\
        电子工程学院 & 大三 \\
        人工智能学院 & 大三 \\
        \bottomrule
    \end{tabular}
\end{center}

网络空间安全学院、现代邮政学院(自动化学院)、理学院、人文学院、数字媒体与设计艺术学院不搬至本部(即在沙河校区读完本科四年)。未来学院是否搬至本部、哪学年搬尚不确定。

除参照上表外,学校已经在\href{https://zsb.bupt.edu.cn/info/1005/1992.htm}{招生章程}中写出了各专业的办学地点。

\faq{沙河校区距离地铁站远吗?是几号线?}

沙河校区附近有两个昌平线地铁站,往南一公里是沙河站,往北七百米是沙河高教园地铁站。沙河校区被这两个站夹在几乎正中间,导致出行不是十分方便。

地铁昌平线沙河站-西土城站区间加上学校和地铁站之间骑车的时间一共需要60-70分钟(不含地铁等车时间),沙河站附近商铺较多,但沙河高教园站乘车人数与沙河站相比会相对较少,实际乘坐时间几乎没有差别。

\faq{有校车吗?免费吗?}

有,校车往返本部和沙河,并且是免费的。学期开始后可以在学校后勤处的公众号“北京邮电大学后勤处”中查看具体班次。校车正常情况下单程用时35-55分钟(不含等车时间)。

沙河校区在图书馆十字路口处靠学生活动中心路北登车,本部在教三楼西侧登车。由于校车要优先服务教师职工,加之高峰时段乘车需求众多,建议提前20分钟到指定地点排队等待。

\faq{寒暑假我可以住在学校吗?}

可以住在学校,但由于本科生从沙河校区搬往本部是在新学年开始时进行,所以下一学年搬往本部的同学,暑假只能住在沙河校区。
